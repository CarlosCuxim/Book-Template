\documentclass[11pt]{article}



% IDIOMA -----------------------------------------------------------------------
\usepackage[spanish, mexico]{babel}

% GRÁFICOS ---------------------------------------------------------------------
\usepackage{geometry}

% MACROS -----------------------------------------------------------------------
\usepackage{packages/qx-tools}

% FUENTE -----------------------------------------------------------------------
\usepackage{newtxtext, newtxmath}
\usepackage[T1]{fontenc}

% HIPERVÍNCULOS ----------------------------------------------------------------
\usepackage{hyperref}


% OTROS ------------------------------------------------------------------------
\title{\textsf{qx-tools}}
\author{Carlos Cuxim\footnote{\url{https://github.com/CarlosCuxim/Exercise-Template}}}
\date{\today}



\begin{document}

\maketitle

\begin{abstract}
  El paquete \textsf{qx-tools} es un conjunto de macros hecho para facilitar la escritura de una actividad de matemáticas. Este se divide en tres paquetes más pequeños. El primer paquete es \textsf{qx-boxes} que define algunas cajas creadas usando \textsf{tcolorbox}\footnote{\url{https://www.ctan.org/pkg/tcolorbox}}. El segundo paquete es \textsf{qx-math} que define múltiples macros matemáticos. Y el último paquete es \textsf{qx-delimiters} que define algunos macros que involucran delimitadores matemáticos. Cada paquete puede ser cargado de manera individual, mientras que el paquete \textsf{qx-tools} carga los tres a la vez.
\end{abstract}




\section{Uso del paquete}

Existen dos formas de usar el paquete. La primera es descargando el código y modificando el archivo \texttt{plantilla.tex}, este archivo ya tiene el paquete instalado, así como un ejemplo de uso. El nombre del archivo se puede cambiar a gusto. Este archivo ya tiene algunos paquetes instalados que son básicos, como \textsf{babel} o \textsf{geometry}, pueden quitarse o agregarse más a gusto, solo hay que mantener la línea de código \verb|\usepackage{packages/qx-tools}|

La segunda forma es ``importando'' únicamente el paquete. Esto se realiza moviendo la carpeta \texttt{packages} a la misma dirección del archivo principal y simplemente agregar en el preambulo \verb|\usepackage{packages/qx-tools}|.

También se puede agregar individualmente cada uno de los paquetes que conforman \textsf{qx-tools}, esto se hace cambiando el comando \verb|\usepackage{packages/qx-tools}| con el nombre del paquete que se desea agregar. Por ejemplo, si solo se quiere el paquete \textsf{qx-delimiters} entonces se agrega \verb|\usepackage{packages/qx-delimiters}| al preámbulo. Lo importante es que nunca se debe omitir el \texttt{packages/} ya que eso le indica a \LaTeX\ que los paquetes están en la carpeta \texttt{packages}.




\section{Paquete \textsf{qx-boxes}}

Este paquete únicamente contiene definiciones de algunas cajas y entornos que pueden ser usados para la redacción de una actividad de matemáticas. A continuación se presentarán todos los tipos de cajas que están definidos en el paquete.


%=========
\paragraph{Caja de ejercicios.} \verb|\begin{exercise} ... \end{exercise}|

\begin{exercise}
  Esta caja está pensada para contener ejercicios. Es una caja simple definida con el paquete \textsf{tcolorbox}.
\end{exercise}


\begin{exercise}
  Cada caja está autonumerada, como si fuera un entorno de teorema creado con \verb|\newtheorem|.
\end{exercise}


\begin{exercise}[$n$-ésimo]
  En el caso de que se quiera una numeración irregular o especial, es posible cambiala manualmente. Basta con simplemente colocar un argumento opcional, por ejemplo, esta caja fue escrita como
    
  \verb|\begin{exercise}[$n$-ésimo] ... \end{exercise}|  
\end{exercise}

\begin{exercise}
  Si se agrega una numeración especial, el contador no cambia. Esto puede ser bueno o malo dependiendo de la situación.
\end{exercise}


\setcounter{qxExercise}{19}
\renewcommand{\theqxExercise}{---\Roman{qxExercise}---}

\begin{exercise}
  Otra forma de modificar el número del ejercicio, para los que sepan un poco más de \LaTeX, es modificando el contador \texttt{qxExercise} o el comando \verb|\theqxExercise|. Por ejemplo, estos comandos fueron agregados antes de esta caja

  \verb|\setcounter{qxExercise}{19}|

  \verb|\renewcommand{\theqxExercise}{---\Roman{qxExercise}---}|
\end{exercise}

\begin{exercise*}
  La versión ``estrellada'' de la caja omite el título y ofrece una forma de obtener una caja simple sin título.

  \verb|\begin{exercise*} ... \end{exercise*}|
\end{exercise*}

\begin{exercise*}[Título chido]
  La versión ``estrellada'' también ofrece una forma de obtener otros títulos, basta con agregar un argumento opcional. Por ejemplo, esta cada fue escrita de la siguiente forma

  \verb|\begin{exercise*}[Título chido] ... \end{exercise*}|
\end{exercise*}


\SetExerciseTitle{Teorema}
\begin{exercise}
  Otra forma de cambiar el título es usando el comando \verb|\SetExerciseTitle|, este cambia el título por default de la caja de ejercicios simple. Por ejemplo, el siguiente comando fue agregado antes de esta caja

  \verb|\SetExerciseTitle{Teorema}|
\end{exercise}


%=========
\paragraph{Cada de definición.} \verb|\begin{definition} ... \end{definition}|

\begin{definition}[Opcional]
  Esta caja está hecha especialmente para definiciones. Funciona igual que un entorno creado con \verb|\newtheorem|, por lo que se puede agregar un texto al lado de la numeración. En específico, esta caja fue creada con el siguiente comando

  \verb|\begin{definition}[Opcional] ... \end{definition}|
\end{definition}

\begin{definition*}[Opcional]
  Al igual con la caja de teorema, este entorno tiene una versión ``estrellada'' el cul permite una definición pero sin la numeración.

  \verb|\begin{definition*}[Opcional] ... \end{definition*}|
\end{definition*}


%=========
\paragraph{Entorno de lemma.}  \verb|\begin{lemma} ... \end{lemma}|

\begin{lemma}
  Para comodidad, ya se han definido el entorno para lemas. Este es un entorno simple definido mediante \verb|\newtheorem|.
\end{lemma}

\begin{lemma*}
  La versión ``estrellada'' omite la numeración, como en el entorno de definición.
\end{lemma*}


%=========
\paragraph{Entorno de solución.} \verb|\begin{answer} ... \end{answer}|

\begin{answer}
  El paquete tiene definido un entorno de solución. Este es simplemente un entorno \verb|{proof}| de \textsf{amsthm}\footnote{\url{https://www.ctan.org/pkg/amsthm}}, pero que dice ``Solución'' y no tiene el cuadrito de demostración.
\end{answer}

\begin{answer}[Cambio de título]
  Al igual que el entorno \verb|{proof}| de \textsf{amsthm}, se puede cambiar la palabra simplemente agregando un argumento opcional.

  \verb|\begin{answer}[Cambio de título] ... \end{answer}|
\end{answer}

\begin{proof}
  Sobra decir que el entorno \verb|{proof}| de \textsf{amsthm} está cargado por defecto en este paquete.
\end{proof}


%=========
\paragraph{Cambio de color.} Es posible cambiar el color de las cajas. Para ello basta con modificar los colores \texttt{qx-box-frame} (modifica el color del borde de las cajas) y \texttt{qx-box-back} (modifica el color interior de las cajas) mediante los comandos del paquete \textsf{xcolor}\footnote{\url{https://www.ctan.org/pkg/xcolor}}.

\colorlet{qx-box-frame}{purple!50!black}
\colorlet{qx-box-back}{yellow!20}

\begin{exercise}
  Los siguientes comandos fueron agregados antes de esta caja.
  
  \verb|\colorlet{qx-box-frame}{purple!50!black}|
  
  \verb|\colorlet{qx-box-back}{yellow!20}|
\end{exercise}

\begin{definition}
  En el caso de la caja de definición, soo se puede modificar el borde, el cual está enlazado al color del borde de la caja de ejercicio.
\end{definition}




\section{Paquete \textsf{qx-math}}

Este paquete únicamente tiene definiciones de comandos de matemáticas. Estos se dividen en símbolos, operadores, entornos matemáticos y comandos auxiliares.

\paragraph{Simbolos y operadores.} El paquete define los siguientes simbolos y operadores.
\begin{center}
\begin{tabular}{ll@{\qquad\qquad}ll}
  $\N$ & \verb|\N| & $\Z$ & \verb|\Z|  \\
  $\Q$ & \verb|\Q| & $\R$ & \verb|\R|  \\
  $\C$ & \verb|\C| & $\bcdot$ & \verb|\bcdot| \\
  $\epsilon$ & \verb|\epsilon| & $\varepsilon$ & \verb|\varepsilon| \\
  $\Re$ & \verb|\Re| & $\Im$ & \verb|\Im| \\
  $\varRe$ & \verb|\varRe| & $\varIm$ & \verb|\varIm|
\end{tabular}
\end{center}

En el caso de \verb|\epsilon| y \verb|\varepsilon|, estos comandos fueron intercambiados. Algo similar sucede con \verb|\Re|, \verb|\Im|, \verb|\varRe| y \verb|\varIm| para conservar los símbolos originales de las partes reales e imaginarias.


\paragraph{Entornos matemáticos.} Actualmente solo están definidos dos entornos. Estos son \verb|{spmatrix}| y \verb|{sbmatrix}| que ofrece una forma más avanzada de los entornos \verb|{pmatrix}| y \verb|{bmatrix}| (respectivamente). So úso es similar al del entorno \verb|{array}|, ya que hay que especificar la cantidad de 
columnas y su alineado, además de que se puede agregar líneas separadoras.
\begin{center}
\begin{minipage}{0.45\textwidth}
\begin{verbatim}\[
  \begin{spmatrix}{c|c}
    A^t & b \\ \hline
    C^t & d
  \end{spmatrix}
    = 
  \begin{sbmatrix}{cc|c}
    a_1 & a_2 & b \\ 
    c_1 & c_2 & d
  \end{sbmatrix}
\]\end{verbatim}
\end{minipage}
\begin{minipage}{0.45\textwidth}
  \[
    \begin{spmatrix}{c|c}
      A^t & b \\ \hline
      C^t & d
    \end{spmatrix}
      = 
    \begin{sbmatrix}{cc|c}
      a_1 & a_2 & b \\ 
      c_1 & c_2 & d
    \end{sbmatrix}
  \]
  \end{minipage}
\end{center}


\paragraph{Comandos auxiliares.} Los comando auxiliares no son más que abreviaciones o funciones útiles para el manejo de ecuaciones.

Las dos que son abreviaciones son \verb|\ds| que es abreviación de \verb|\displaystyle| y \verb|\bec| que es abreviación de \verb|\mathbf|.

En cuanto a las funciones útiles, el primer tipo son conectores de ecuaciones, que permiten colocar ``y'' y ``o'' a una cierta distancia. Los comandos \verb|\eqand| y \verb|\eqor| agregan una separación de un \verb|\quad|, mientras que  \verb|\Eqand| y \verb|\Eqor| agregan una separación de un \verb|\qquad|.
\begin{center}
  \begin{minipage}{0.45\textwidth}
  \begin{verbatim}\[ A \eqand B \]
\[ A \Eqand B \]
\[ A \eqor B \]
\[ A \Eqor B \]\end{verbatim}
  \end{minipage}
  \begin{minipage}{0.45\textwidth}
    \[ A \eqand B \]
    \[ A \Eqand B \]
    \[ A \eqor B \]
    \[ A \Eqor B \]
  \end{minipage}
\end{center}

Por último tenemos el comando \verb|\tagthis| que agrega el numerado a una ecuacion específica. Este comando es bastante útil si se usa con el entorno \verb|{align*}|.
\begin{center}
  \begin{minipage}{0.45\textwidth}
  \begin{verbatim}\begin{align*}
  a &= b, \\
  b &= c, \\
  c &= d, \tagthis \\
  d &= e.
\end{align*}\end{verbatim}
  \end{minipage}
  \begin{minipage}{0.45\textwidth}
    \begin{align*}
      a &= b, \\
      b &= c, \\
      c &= d, \tagthis \\
      d &= e.
    \end{align*}
  \end{minipage}
\end{center}



\section{Paquete \textsf{qx-delimiters}} Este paquete es el más pequeño. Lo que hace es definir únicamente comandos para delimitadores matemáticos.

\paragraph{Parentesis, corchetes y llaves.} Los primeros comandos de delimitadores que ofrece el paquete son \verb|\paren| (para paréntesis), \verb|\corch| [para corchetes] y \verb|\set| \{para llaves\}. Estos delimitadores aunque crecen, difieren ligeramente de si usaras \verb|\left| y \verb|\right|. La principal diferencia con el uso de \verb|\left| y \verb|\right| es que ofrecen un mejor espaciado en algunas situaciones, como cuando son usados como argumento de una función.

\begin{center}
  \begin{minipage}{0.45\textwidth}
  \begin{verbatim}\[
  \paren{\frac{a}{b}}, \qquad
  \corch{\frac{a}{b}}, \qquad
  \set{\frac{a}{b}}.
\]

Diferencia con \verb|\left|
y \verb|\right|
\[
  f\paren{\frac{a}{b}}
    \qquad\text{vs}\qquad
  f\left(\frac{a}{b}\right)
\]\end{verbatim}
  \end{minipage}
  \begin{minipage}{0.45\textwidth}
  \[
    \paren{\frac{a}{b}}, \qquad
    \corch{\frac{a}{b}}, \qquad
    \set{\frac{a}{b}}.
  \]

  Diferencia con \verb|\left| y \verb|\right|
  \[
    f\paren{\frac{a}{b}}
      \qquad\text{vs}\qquad
    f\left(\frac{a}{b}\right)
  \]
  \end{minipage}
\end{center}

La diferencia de usar estos comandos comparados con el uso de \verb|\left| y \verb|\right| es mínima y puede que algunos no lo noten (se podría considerar que estos comandos son un capricho mío). Además, aunque diga que tiene un espaciado mejor, creo que hay algunas situaciones donde es mejor usar \verb|\left| y \verb|\right|, como cuando se define un conjunto o los paréntesis se usan para simbolizar multiplicación de fórmulas grandes. El uso de uno u otro no es una regla exacta y depende de cada caso. Si no te importan estas cuestione técnicas puedes usar el que más te guste.

\paragraph{Valor absoluto, norma, producto interno, piso y techo.} Como el nombre lo indica, estos son los delimitadores para el valor absoluto \verb|\abs| y \verb|\Abs|; producto interno (o comillas angulares) \verb|\inner| y \verb|\Inner|; norma \verb|\norm| y \verb|\Norm|; piso \verb|\floor| y \verb|\Floor|; y techo \verb|\ceil| y \verb|\Ceil|.

Antes de pasar a los comandos en sí, hay que explicar por qué tienen dos versiones, una minúscula y otra mayúscula. La versión minúscula ofrece una versión que no crece mientras que la mayúscula ofrece una versión que si crece. El motivo es que existe un error en \LaTeX\ que hace que algunos símbolos, como \texttt{|}, sean tratados como símbolos ordinarios (como si fueran letras, pues) en vez de un delimitador (son cosas algo tecnicas, una disculpa), lo que hace que el espaciado a veces falle. Por ejemplo en $|-x|$, que el correcto espaciado sería $\abs{-x}$. De este modo, las versiones minúsculas son una forma de obtener el correcto espaciado para delimitadores que no crecen.

Una pregunta válida sería ¿por qué no usar la versión que crece siempre? La respuesta es que a veces se necesita que no crezca el delimitador, como en $\abs{x^2}$ que si se usara la versión que crece se vería así $\Abs{x^2}$, que aunque no parezca que esté mal es tipográficamente incorrecto\footnote{Para más información sobre la ortotipografía matemática puedes consultar el siguiente artículo de Javier Bezos \url{http://www.texnia.com/archive/ortomatem.pdf}, lo recomiendo leer si quieres documentos bonitos.} (te digo que todo este paquete es un capricho mío). Si a ti no te importa estas cuestiones muy técnicas puedes usar siempre la versión mayúscula. En el futuro puede que haga una opción para que las versiones minúsculas sean las que crezcan.

Con toda esta biblia dicha acerca de para que hay dos comandos, ahora si podemos pasar a ver los comandos como tal.

\begin{center}
  \begin{minipage}{0.45\textwidth}
  \begin{verbatim}\[
  \abs{x^2}, \qquad
  \Abs{\frac{a}{b}}
\]
\[
  \norm{x^2}, \qquad
  \Norm{\frac{a}{b}}
\]
\[
  \inner{x^2}, \qquad
  \Inner{\frac{a}{b}}
\]
\[
  \floor{x^2}, \qquad
  \Floor{\frac{a}{b}}
\]
\[
  \ceil{x^2}, \qquad
  \Ceil{\frac{a}{b}}
\]\end{verbatim}
  \end{minipage}
  \begin{minipage}{0.45\textwidth}
  \[
    \abs{x^2}, \qquad
    \Abs{\frac{a}{b}}
  \]
  \[
    \norm{x^2}, \qquad
    \Norm{\frac{a}{b}}
  \]
  \[
    \inner{x^2}, \qquad
    \Inner{\frac{a}{b}}
  \]
  \[
    \floor{x^2}, \qquad
    \Floor{\frac{a}{b}}
  \]
  \[
    \ceil{x^2}, \qquad
    \Ceil{\frac{a}{b}}
  \]
  \end{minipage}
\end{center}

Una aclaración es que los comandos que crecen tienen el espaciado correcto, como los comandos para paréntesis, corchetes y llaves, por lo que sí difieren de usar \verb|\left| y \verb|\right|.


\paragraph{Delimitador intermedio} Finalmente hay un comando más. Este es \verb|\relmiddle| hace la misma operación que \verb|\middle|, pero con un mejor espaciado (en específico el espaciado de una relación binaria). Funciona tanto con \verb|\left| y \verb|\right| como con los comandos del paquete que crecen.
\begin{center}
  \begin{minipage}{0.45\textwidth}
  \begin{verbatim}\[
  \left\{ \prod a_i \relmiddle|
    i \in X \right\}
\]
\[
  \paren{ \frac{a}{b} \relmiddle/
    \frac{a}{b} }
\]
\[
  \Inner{ \sum a_i \relmiddle\|
  \sum b_i }
\]\end{verbatim}
  \end{minipage}
  \begin{minipage}{0.45\textwidth}
  \[
    \left\{ \prod a_i \relmiddle|
      i \in X \right\}
  \]
  \[
    \paren{ \frac{a}{b} \relmiddle/
      \frac{a}{b} }
  \]
  \[
    \Inner{ \sum a_i \relmiddle\|
    \sum b_i }
  \]
  \end{minipage}
\end{center}

De igual forma, este comando no sustituya al \verb|\middle| sino que es un complemento de este. Dependiente de la situación se debería usar uno u otro.

\end{document}