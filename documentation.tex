\documentclass[11pt]{article}



% IDIOMA -----------------------------------------------------------------------
\usepackage[spanish, mexico]{babel}

% GRÁFICOS ---------------------------------------------------------------------
\usepackage{geometry}

% MACROS -----------------------------------------------------------------------
\RequirePackage{packages/qx-tools}

% FUENTE -----------------------------------------------------------------------
\usepackage{newtxtext, newtxmath}
\usepackage[T1]{fontenc}



% OTROS ------------------------------------------------------------------------
\title{\textsf{qx-tools}}
\author{Carlos Cuxim}
\date{\today}



\begin{document}

\maketitle

\begin{abstract}
  El paquete \textsf{qx-tools} es un conjunto de macros hecho para facilitar la escritura de un documento matemático. Este se divide en tres paquetes más pequeños. El primer paquete es \textsf{qx-boxes} que define algunas cajas creadas usando \textsf{tcolorbox}. El segundo paquete es \textsf{qx-math} que define múltiples macros matemáticos. Y el último paquete es \textsf{qx-delimiters} que define algunos macros que involucran delimitadores matemáticos. Cada paquete puede ser cargado de manera individual, mientras que el paquete \textsf{qx-tools} carga los tres a la vez.
\end{abstract}



\section{Instalación}


\section{\textsf{qx-boxes}}



\end{document}