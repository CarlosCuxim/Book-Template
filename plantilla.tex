\documentclass[draft]{qx-files/qx-book}





% TEMP -------------------------------------------------------------------------
\usepackage{babel}
\usepackage{lipsum}
% TEMP -------------------------------------------------------------------------






% OTROS ------------------------------------------------------------------------
\title{Plantilla}
\author{Nombre de los autores}
\date{\today}



\begin{document}

\frontmatter


\maketitle





\chapter{Prefacio}

\lipsum[1]




\tableofcontents

\mainmatter

\part{Una de cal}


\chapter[Álgebra conmutativa]{Algunos tópicos en álgebra conmutativa}
\lipsum[2]
\begin{equation}
  x^2 + y^2 = 0
\end{equation}

\lipsum[1]

\section{Las patatas}

\lipsum[1]



\lipsum[1]

\section{Las uvas}

\lipsum[1]

\subsection{Las verdes}

\lipsum[1]

\chapter*{Sin nada voy a hacer esta wea muy larga para ver como se comporta esta wea}

\subsection{Las rojas}

\lipsum[1]

\section{Una vía de doble visión}

\lipsum

\chapter[Título largo]{Título innecesariamente largo para probar como se comporta en esta wea}
\lipsum


\part{Dos de arena}

\chapter{Tres}
\lipsum


\chapter{Cuatro}
\lipsum[1]

\appendix

\chapter{Apéndice primero}


\lipsum[1]

\chapter{Apéndice segundo}


\lipsum[1]

\chapter{Apéndice tercero}


\lipsum[1]

\backmatter

\chapter{Bibliografía}

\lipsum

\end{document}

