\documentclass[11pt]{article}



% IDIOMA -----------------------------------------------------------------------
\usepackage[spanish, mexico]{babel}

% GRÁFICOS ---------------------------------------------------------------------
\usepackage{geometry}

% MACROS -----------------------------------------------------------------------
\usepackage{packages/qx-tools}

% FUENTE -----------------------------------------------------------------------
\usepackage{lmodern} % Se puede cambiar por la fuente con la que se sienta más a gusto
\usepackage[T1]{fontenc}



% OTROS ------------------------------------------------------------------------
\title{Plantilla}
\author{Nombre de los autores}
\date{\today}



\begin{document}

\maketitle


\section{Entornos}

Existen tres entornos definidos cada una con una versión alterna.

\begin{exercise}
  Entorno \verb|{exercise}|. Se autonumera.
\end{exercise}

\begin{exercise*}
  Entorno \verb|{exercise*}|. Es un auxiliar de la caja anterior.
\end{exercise*}

Se puede cambiar el título de ``Ejercicio'' con el comando \verb|\SetExerciseTitle| en algún lugar del documento (antes de la caja). También se puede poner una numeración personalizada colocando un parámetro opcional (con corchetes), si se deja vacío no se numera. 


\SetExerciseTitle{Problema}

\begin{exercise}[(Numeración especial)]
  Caja de ejercicios con numeración especial y \verb|\SetExerciseTitle{Problema}|.
\end{exercise}

\begin{exercise}[]
  Caja de ejercicios con la numeración vacía y \verb|\SetExerciseTitle{Problema}|.
\end{exercise}



\begin{definition}[Texto opcional]
  Entorno \verb|{definition}|. Se autonumera y se puede poner un texto opcional.
\end{definition}

\begin{definition*}
  Entorno \verb|{definition*}|. Sin numeración.
\end{definition*}

\begin{lemma}
  Entorno \verb|{lema}|. Se autonumera, es un entorno creado con \verb|\newtheorem|.
\end{lemma}

\begin{lemma*}
  Entorno \verb|{lema*}|. Sin numeración.
\end{lemma*}




\end{document}

